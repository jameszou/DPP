\documentclass[idxtotoc,hyperref,openany]{labbook} % 'openany' here removes the gap page between days, erase it to restore this gap; 'oneside' can also be added to remove the shift that odd pages have to the right for easier reading

\usepackage[ 
  backref=page,
  pdfpagelabels=true,
  plainpages=false,
  colorlinks=true,
  bookmarks=true,
  pdfview=FitB]{hyperref} % Required for the hyperlinks within the PDF
  
\usepackage{booktabs} % Required for the top and bottom rules in the table
\usepackage{float} % Required for specifying the exact location of a figure or table
\usepackage{graphicx} % Required for including images
\usepackage{lipsum} % Used for inserting dummy 'Lorem ipsum' text into the template

\newcommand{\HRule}{\rule{\linewidth}{0.5mm}} % Command to make the lines in the title page
\setlength\parindent{0pt} % Removes all indentation from paragraphs

\begin{document}

\labday{Monday, September 10, 2012 }

\section{Dynamic programming and DPP}
For concreteness, we focus on Nussinov's DP algorithm. The ideas should work for general DPs.  Suppose we have a fixed DNA/RNA sequence $S$ of length $L$. Let $y$ denote a particular secondary folding structure of the sequence. We try to follow the notations of the Supplementary Methods of Taskar's SDPP paper. 

The factor graph has $T=L(L-1)$ number of variable nodes, indexed by $t=(i,j), i<j$. The possible values for the variable node $(i,j)$ is the set of energies possible for the subsequence $S(i,j)$. There are also $L(L-1)$ factor nodes indexed by $(i,j), i<j$. The factor node $F(i,j)$ is connected to the variable nodes $\{(i+1, j-1), (i+k,j), (k+1, j)\}, i<k<j $. For $(i,j) \neq (1, L)$, the factor $F(i,j) = 1$ if the substructure $y(i,j)$ has a consistent folding (no mismatched base-pairs, etc), and $F(i,j)=0$ if the substructure is not consistent. For the "last" factor, $F(1,L)=\mbox{energy of the structure y}$.




\end{document}