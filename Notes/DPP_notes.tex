\documentclass[idxtotoc,hyperref,openany]{labbook} % 'openany' here removes the gap page between days, erase it to restore this gap; 'oneside' can also be added to remove the shift that odd pages have to the right for easier reading

\usepackage[ 
  backref=page,
  pdfpagelabels=true,
  plainpages=false,
  colorlinks=true,
  bookmarks=true,
  pdfview=FitB]{hyperref} % Required for the hyperlinks within the PDF
  
\usepackage{booktabs} % Required for the top and bottom rules in the table
\usepackage{float} % Required for specifying the exact location of a figure or table
\usepackage{graphicx} % Required for including images
\usepackage{lipsum} % Used for inserting dummy 'Lorem ipsum' text into the template

\newcommand{\HRule}{\rule{\linewidth}{0.5mm}} % Command to make the lines in the title page
\setlength\parindent{0pt} % Removes all indentation from paragraphs

\begin{document}

\labday{Monday, September 10, 2012 }

\section{Dynamic programming and DPP}
For concreteness, we focus on Nussinov's DP algorithm. The ideas should work for general DPs.  Suppose we have a fixed DNA/RNA sequence $S$ of length $L$. Let $y$ denote a particular secondary folding structure of the sequence. We try to follow the notations of the Supplementary Methods of Taskar's SDPP paper. 

The factor graph has $T=L(L-1)$ number of variable nodes, indexed by $t=(i,j), i<j$. The possible values for the variable node $(i,j)$ is the set of energies possible for the subsequence $S(i,j)$. There are also $L(L-1)$ factor nodes indexed by $(i,j), i<j$. The factor node $F(i,j)$ is connected to the variable nodes $\{(i+1, j-1), (i+k,j), (k+1, j)\}, i<k<j $. For $(i,j) \neq (1, L)$, the factor $F(i,j) = 1$ if the substructure $y(i,j)$ has a consistent folding (no mismatched base-pairs, etc), and $F(i,j)=0$ if the substructure is not consistent. For the "last" factor, $F(1,L)=\mbox{energy of the structure y}$.

\labday{Saturday, September 22, 2012}

To fill in a bit more and correct some mistakes from above.

We have a factor graph with $T=\frac{L(L-1)}{2}$ variable nodes, indexed by $t=(i,j)$. Similarly, there are $\frac{L(L-1)}{2}$ factor nodes that we denote by $F(i,j)$. The factor node  $F(i,j)$ is connected to the variable nodes $\{(i+1, j-1), (k,j), (i,k)\}, i<k<j $.  Generic variable nodes are denoted by $t$, and generic factor notes are denoted by $\alpha$. 

\begin{enumerate}
\item The values for $y_t$ are the possible folding scores for the subsequence $S(i,j)$. In the simplest case, the folding score equals to the number of base pairings, and $y_t \in \{0, 1, ..., L/2\}$.
\item For a factor node $\alpha$, we think of $y_{\alpha}$ as a list of values $\{y_{t_1}, y_{t_2}, ...\}$ where $t_i$ is a variable node connected to $\alpha$. Let $\alpha*$ denote the final factor node $F(1,L)$.
\item A factor node is associated with weight $w_{\alpha}(y_{\alpha})$. We have 
\[
w_{\alpha}(y_{\alpha})=(q^2(y_{\alpha}), q^2(y_{\alpha})\phi_r(y_{\alpha}), q^2(y_{\alpha})\phi_l(y_{\alpha}), q^2(y_{\alpha})\phi_r(y_{\alpha})\phi_l(y_{\alpha}) )
\]
\item For $\alpha \neq \alpha*$, $q(y_{\alpha})=1$ if the list $y_{\alpha}$ is feasible, i.e. the score $y_{ij}$ can be feasibly arrived from $y_{(i+1)(j-1)}, y_{kj}, y_{ik}$ and the sequence $S(i,j)$. And $q(y_{\alpha})=0$ otherwise. For $\alpha*$, $y_{\alpha*}=y_{1L}$, the score of the entire structure. 
\end{enumerate}

Questions to address:
\begin{enumerate}
\item The two pass belief propagation computes $\sum_{y} \prod_{\alpha} w_{\alpha}(y_{\alpha})$, where $y=\{y_{ij}\}$ is a list of scores over all substructures. What we actually need to compute is $\sum_{s} \prod_{\alpha} w_{\alpha}(y_{s})$, where the sum is over all feasible substructures. Seems like there is an straightforward bijection between $y$ and the set of all possible structures for S. Need to verify.
\item For the two pass belief propagation to work, i.e. equation 3 of the supplement, we need the factor graph to be a tree. This is not true in our case, which has many loops. However, the weights at the factor nodes are simple, $w_{\alpha}(y_{\alpha}=(0, 0, 0, 0))$ if $y_{\alpha}$ is not feasible. So might still work.
\item As we discussed, the factor graph is densely connected. But for a given structure, only a few variable nodes $y_t$ contribute. Need to work this out precisely.
\end{enumerate}

\end{document}